\chapter*{Introduction}
\addcontentsline{toc}{chapter}{Introduction}
Fluid-structure interactions (FSI) describe situations in which a movable or deformable object interacts with surrounding or internal fluid flow. 
It can have a stable or oscillatory nature, depending on the problem. FSI problems can be found in engineering e.g. aircraft, automobile, spacecraft, or engine design,
and civil engineering e.g. when designing bridges or huge marine structures. FSI problems also arise in biomechanics when modelling blood flow or when studying heart
valve dynamics.

These problems are most of the time too complex to be solved analytically and it is often impossible to conduct relevant experiments. 
If that is the case we need to analyse these problems utilising numerical simulation. Two main approaches for FSI are being used:
\begin{itemize}
    \item Partitioned approach: the governing equations for fluid and solid are solved separately, with two distinct solvers, and coupling is achieved subsequently.
    \item Monolithic approach: the governing equations for both fluid and solid are solved simultaneously, with a single solver.
\end{itemize} 
The partitioned approach often allows us to use existing software which might be advantageous. However, the stability of the coupling algorithm might be an issue. 
Moreover working with a mesh that changes over time is often cumbersome and hard to implement if the geometry of the problem is complex.

We turn our attention to a monolithic approach; specifically the Smoothed Particle Hydrodynamics (SPH) method that was developed in [1] by discretizing the unified formulation of continuum mechanics [2,3].
The governing equations belong to the class of Symmetric Hyperbolic Thermodynamically compatible (SHTC) equations where we substitute the concept of viscosity coefficient by the so-called
particle rearrangement time. The continuum is interpreted as a system of particles connected by bonds and flow is interpreted as bond destruction and rearrangement of particles. 
This is the key idea for a unified description of fluids and solids because (elastic) solids can be interpreted as fluids with infinite relaxation time.

The aim of this paper is to expand on results achieved in paper [1] and test the method for FSI problems. First, we review article [1], the SPH method and we derive the governing equations
from hamiltonian mechanics of continuum. We then proceed to numerical testing namely we test the method for viscoplastic flows. Finally, we put the method in test on the Turek-Hron FSI benchmark.

