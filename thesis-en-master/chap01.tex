\chapter{Title of the first chapter}

An~example citation: \cite{Andel07}

\section{Title of the first subchapter of the first chapter}
The thesis should consist of the following steps:
1 Review of paper [1], where an SPH method for both fluids and solids was developed.
2 Review of the SPH method and simple numerical tests using SmoothedParticles.jl framework [2].
3 Review of fluid-structure interaction (FSI) problems in continuum mechanics, for instance the Turek-Hron benchmark [3].
4 An attempt to simulate an FSI problem using SPH.


Abstract:
Smoothed Particle Hydrodynamics (SPH for short) is lagrangian meshfree method that is advantageous 
for simulations of fluids in domain with free boundary. The method in use is based on discretization
of the Symmetric Hyperbolic Thermodynamically Compatible equations (SHTC), that can describe both fluids, elastic solids,
and visco-elasto-plastic solids withing a single framework. Firstly, we review this method. Secondly, we test it on Turek-Hron benchmark and we show that
the method is suitable for simulating FSI (fluid structure interactions) problems and shows remarkable agreement with the data.

Introduction:pr
Something about SPH
Hamiltonian formulation, Brackets

Smoothed Particle Hydrodynamics is a particle based lagrangian meshfree method for solving partial differential equations. 
It is especially useful when dealing with free surfaces. SPH also conserves energy, linear and angular momentum. 
Among the disadvantages are slower convergence rates and numerical artifacts such as particle clumping or tensile instability [cite].


Fluid-structure interaction (FSI) describes situation where some movable or deformable object interacts with surrounding or internal fluid flow. 
It can have stable or oscillatory nature, depending on the problem. FSI problems can be found in engineering e.g. aircraft, automobile, spacecraft or engine design,
in civil engineering e.g. when designing bridges or huge marine structures. FSI problems also arise in biomechanics when modelling blood flow or when studying heart
valve dynamics.
These problems are often too complex to be solved analyticaly and it is often impossible to conduct relevant experiments. 
If that is the case we need to analyze these problems by means of numerical simulation. Two main aproaches for FSI are being used:
 - Partitioned approach: the governing equations for fluid and for solid are solved separately, with two distinct solvers, coupling is achieved subsequently.
 - Monolithic approach: the governing equations for both fluid and solid are solved simultaneously, with a single solver.
The partitioned approach often allows us to use existing software which might be advantageous. However the stability of the coupling algorithm migh be an issue. 
Moreover working with a mesh that changes over time is often cumbersome and hard to implement if the geometry of the problem is complex.
We turn our attention to monolithic approach specifically the Smoothed Particle Hydrodynamics (SPH) method that was developed in [1] by discretizing the unified formulation of continuum mechanics [2,3].
The governing equations belong to the class of Symmetric Hyperbolic Thermodynamically compatible (SHTC) equations where we subsitute the concept of viscosity coefficient by the so-called
particle rearrangement time. The continuum is interpreted as a system of of particles connected by bonds and flow is interpreted as bond destruction and rearrangement of particles. 
This is the key idea for unified description of fluids and solids because (elastic) solids can be interpreted as fluids with infinite relaxation time.
The aim of this paper is to expand on results achieved in paper [1] and test the method for FSI problems. First, we review article [1], the SPH method and we derive the governing equations
from hamiltonian mechanics of continuum. We then proceed to some examples and we namely test the method for viscoplastic flows. Finally, we put the method in test on the Turek-Hron FSI benchmark.
1. Introduction

The introduction should outline the following elements:

    Background: Describe the significance of fluid-structure interaction (FSI) problems in computational physics and engineering,
    highlighting their relevance across fields such as biomechanics, aerospace, and civil engineering.

    Current Challenges: Discuss the limitations of traditional approaches in simulating both fluids and solids,
    especially in contexts where interactions between the two are critical. Explain how typical FSI models often require
     separate formulations and interactions for fluid and solid domains.

    Objective: State the purpose of your research: developing a unified framework for FSI that leverages the SHTC
    equations for simultaneous treatment of fluid and solid mechanics within a single SPH-based model.

    Contribution: Summarize the unique contribution of this study, specifically the use of the SHTC
    equations discretized with SPH and the demonstration of the model on the Turek-Hron benchmark.



Derivation and the choice of governing equations. Discretization of those equations.
Review of the SPH method and simulation for viscous flow. 
$\Id$
$\dive \pd$
\section{Title of the second subchapter of the first chapter}
