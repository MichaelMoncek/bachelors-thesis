\chapter{Title of the second chapter}

\section{SPH}
Smoothed-Particle Hydrodynamics (SPH) is a lagrangian method that works by dividing the continuum into N particles. 
Each particle $a$ has its position $\xx_a$, velocity $\vv_a$, mass $m_a$, density $\rho_a$ etc. Given the lagrangian
structure of SPH it follows that $\dot{\xx}_a = \vv_a$.
Particles interact with each other via the $ smoothing$ $kernel$ function $W_h : \R^d \goto [0, \infty)$. Parameter $h > 0$ is called
$smoothing$ $length$. Kernel function has the following properties:
\begin{itemize}
    \item $W_h \in C^2(\R^d)$
    \item $\int_{\R^d} W_h(\xx) \,d\xx = 1$
    \item $W_h = W_h(|\xx|)$
    \item $h: W_h(r) = \frac{1}{h^d}W_1(r/h)$
    \item $\frac{dW_h}{dr} \leq 0$
\end{itemize}
In this thesis, we will be using Wendland's quintic kernel, which is defined as follows:
\begin{equation}
    W_h(r) =
    \begin{cases}
        \frac{\alpha_d}{h^d} (1 - \frac{r}{2h})^4(1 + \frac{4r}{h})
        & 0 \leq r \leq 2h,\\
        0
        & 2h < r.
    \end{cases} 
\end{equation}
The constant 
\begin{equation}
    \alpha_d =
    \begin{cases}
        \frac{7}{4}\pi &d = 2,\\
        \frac{21}{16}\pi &d = 3
    \end{cases}
\end{equation}
is choosen such that the normalization constraint $\int_{\R^d} W_h(\xx) \,d\xx = 1$ is satisfied.
Functions and partial derivatives are aproximated via the smoothing kernel in two steps. 

Firstly we aproximate $f$ 
respectively ${\pd}_x f$ using convolution:
\begin{equation}
    \begin{aligned}
            f = f * \delta &\approx f * W_h, \\
            {\pd}_x f = {\pd}_x f * \delta &\approx {\pd}_x f * W_h = f * {\pd}_x W_h.    
    \end{aligned}
\end{equation}

Secondly, we aproximate the integral by discreticazing it and summing over all particles in the system:
\begin{equation}
    \begin{aligned}
            (f * W_h)({\xx}_a) = \int f(\mathbf{y}) W_h({\xx}_{a} - \mathbf{y}) d\mathbf{y} \approx \sum_{b}  V_b f_b W_{h, ab}, \\
            (f * {\pd}_{x_i} W_h)({\xx}_a) = \int f(\mathbf{y}) {\pd}_{x_i} W_h({\xx}_a - \mathbf{y}) d\mathbf{y} \approx \sum_{b}  V_b f_b {\pd}_{x_i} W_{h, ab}
    \label{eq:aproxsph}           
    \end{aligned}
\end{equation}
Where $V_b = \rho_b / m_b$ is the volume of the b-th particle, $f_b = f({\xx}_b)$ and $W_{h, ab} = W_h({\xx}_a - {\xx}_b)$.
If there is no room for disambiguation, we will ommit index $h$ and simply write $W_{ab} = W_{h, ab}$. We will use the notation $W'_{ab} = \frac{dW}{dr}(r_{ab})$
and $\nabla W_{ab} = W'_{ab} \frac{{\xx}_{ab}}{r_{ab}}$ where $r_{ab} = |{\xx}_a - {\xx}_b| = |{\xx}_{ab}|$.


\subsection{Discrete density}
The density $\rho_a$ of particle $a$ can be expressed as: 
\begin{equation}
    \rho_a = \rho({\xx}_a) = \sum_{b} m_b W_{ab}.
\end{equation}
It follows that the total differential of $\rho_a$ with respect to the position of the particles $\xx$ is equal to:
\begin{equation}
    d\rho_a = \sum_{b} m_b dW_{ab} = \sum_{b} m_b \sum_{b} \frac{dW_{ab}}{dr} = \sum_{b} \nabla W_{ab} \cdot d{\xx}_{ab}.
\end{equation} 
assuming $r_{ab} > 0$ for each pair of particles $a$ and $b$ and using the fact that for vector $\xx = (x_1,...,x_d)^T$ we have:
\begin{equation}
    dW = \sum_{i=1}^{d} \frac{dW}{dr} \frac{dr}{dx_i} dx_i =  W' \frac{\xx}{r} \cdot d\xx = \nabla W \cdot d\xx
\end{equation}

\subsection{Discrete divergence and gradient}
%We can aproximate divergence $\nabla \cdot \vv$ of vector $\vv = (v_1,..., v_d)^T$ and using ~\Ref{eq:aproxsph} as follows
%\begin{equation}
%    \nabla \cdot \vv = \sum_{i=1}^d \frac{dv_i}{dx_i} = \sum_{i=1}^{d} \sum_{b}  V_b f_b {\pd}_{x_i} W_{h, ab}
%\end{equation}
Assuming all particles have constant mass and differentiating the discrete density $\rho_a$ with respect to time yields
\begin{equation}
    \frac{d\rho_a}{dt} = \sum_b m_b \frac{dW_{ab}}{dt} = \sum_b m_b \nabla W_{ab} \cdot \frac{d{\xx}_{ab}}{dt} = \sum_b m_b {\vv}_{ab} \cdot \nabla W_{ab}
\end{equation}
Comparing the result with continuity equation in the Lagrangian setting 
\begin{equation}
    \frac{d\rho}{dt} = - \rho \nabla \cdot \vv
\end{equation}
motivates us to define the \emph{discrete divergence} operator as follows
\begin{equation}
    (\nabla \cdot \vv)_a = - \frac{1}{\rho_a}\sum_b m_b {\vv}_{ab} \cdot \nabla W_{ab}.
\label{eq:sphdiv}
\end{equation}
Similarly for gradient $\nabla {\vv}$ we have
\begin{equation}
    (\nabla {\vv}_a) = - \frac{1}{\rho_a}\sum_b m_b {\vv}_{ab} \otimes \nabla W_{ab}.
\label{eq:sphgrad}
\end{equation}

\subsection{Discrete velocity gradient}
The velocity gradient $\LL$ satisfies the following equation
\begin{equation}
    \dot{\FF} = \LL \FF
\end{equation}
where $\dot{\FF} = \pd \FF + \vv \cdot \nabla \FF$ is the material derivative of $\FF$ and $\FF = \frac{\pd \xx}{\pd \XX}$ is the deformation gradient.
Using \Ref{eq:sphgrad} we can aproximate $\LL$ as follows
\begin{equation}
    {\LL}_a = \left(\sum_{b}m_b{\vv}_{ab}\otimes \nabla W_{ab}\right) \left(\sum_{b}m_b{\xx}_{ab}\otimes \nabla W_{ab}\right)^{-1}
\end{equation}
\section{Governing equations}
The unified model of continuum for both fluid and solid dynamics reads as follows [cite]:
\begin{subequations}
    
\end{subequations}

