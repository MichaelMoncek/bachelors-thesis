%%% The main file. It contains definitions of basic parameters and includes all other parts.

% Meta-data of your thesis (please edit)
\input metadata.tex

% Generate metadata in XMP format for use by the pdfx package
\input xmp.tex

%% Settings for single-side (simplex) printing
% Margins: left 40mm, right 25mm, top and bottom 25mm
% (but beware, LaTeX adds 1in implicitly)
\documentclass[12pt,a4paper]{report}
\setlength\textwidth{145mm}
\setlength\textheight{247mm}
\setlength\oddsidemargin{15mm}
\setlength\evensidemargin{15mm}
\setlength\topmargin{0mm}
\setlength\headsep{0mm}
\setlength\headheight{0mm}
% \openright makes the following text appear on a right-hand page
\let\openright=\clearpage

%% Settings for two-sided (duplex) printing
% \documentclass[12pt,a4paper,twoside,openright]{report}
% \setlength\textwidth{145mm}
% \setlength\textheight{247mm}
% \setlength\oddsidemargin{14.2mm}
% \setlength\evensidemargin{0mm}
% \setlength\topmargin{0mm}
% \setlength\headsep{0mm}
% \setlength\headheight{0mm}
% \let\openright=\cleardoublepage

%% If the thesis has no printed version, symmetric margins look better
% \documentclass[12pt,a4paper]{report}
% \setlength\textwidth{145mm}
% \setlength\textheight{247mm}
% \setlength\oddsidemargin{10mm}
% \setlength\evensidemargin{10mm}
% \setlength\topmargin{0mm}
% \setlength\headsep{0mm}
% \setlength\headheight{0mm}
% \let\openright=\clearpage

%% Generate PDF/A-2u
\usepackage[a-2u]{pdfx}

%% Prefer Latin Modern fonts
\usepackage{lmodern}
% If we are not using LuaTeX, we need to set up character encoding:
\usepackage{iftex}
\ifpdftex
\usepackage[utf8]{inputenc}
\usepackage[T1]{fontenc}
\usepackage{textcomp}
\fi

%% Further useful packages (included in most LaTeX distributions)
\usepackage{amsmath}        % extensions for typesetting of math
\usepackage{amsfonts}       % math fonts
\usepackage{amsthm}         % theorems, definitions, etc.
\usepackage{bm}             % boldface symbols (\bm)
\usepackage{booktabs}       % improved horizontal lines in tables
\usepackage{caption}        % custom captions of floating objects
\usepackage{dcolumn}        % improved alignment of table columns
\usepackage{floatrow}       % custom float environments
\usepackage{graphicx}       % embedding of pictures
\usepackage{indentfirst}    % indent the first paragraph of a chapter
\usepackage[nopatch=item]{microtype}   % micro-typographic refinement
\usepackage{paralist}       % improved enumerate and itemize
\usepackage[nottoc]{tocbibind} % makes sure that bibliography and the lists
			    % of figures/tables are included in the table
			    % of contents
\usepackage{xcolor}         % typesetting in color

% The hyperref package for clickable links in PDF and also for storing
% metadata to PDF (including the table of contents).
% Most settings are pre-set by the pdfx package.
\hypersetup{unicode}
\hypersetup{breaklinks=true}

% Packages for computer science theses
\usepackage{algpseudocode}  % part of algorithmicx package
\usepackage{algorithm}
\usepackage{fancyvrb}       % improved verbatim environment
\usepackage{listings}       % pretty-printer of source code

% You might want to use cleveref for references
% \usepackage{cleveref}

% Set up formatting of bibliography (references to literature)
% Details can be adjusted in macros.tex.
%
% BEWARE: Different fields of research and different university departments
% have their own customs regarding bibliography. Consult the bibliography
% format with your supervisor.
%
% The basic format according to the ISO 690 standard with numbered references
\usepackage[natbib,style=iso-numeric,sorting=none]{biblatex}
% ISO 690 with alphanumeric references (abbreviations of authors' names)
%\usepackage[natbib,style=iso-alphabetic]{biblatex}
% ISO 690 with references Author (year)
%\usepackage[natbib,style=iso-authoryear]{biblatex}
%
% Some fields of research prefer a simple format with numbered references
% (sorting=none tells that bibliography should be listed in citation order)
%\usepackage[natbib,style=numeric,sorting=none]{biblatex}
% Numbered references, but [1,2,3,4,5] is compressed to [1-5]
%\usepackage[natbib,style=numeric-comp,sorting=none]{biblatex}
% A simple format with alphanumeric references:
%\usepackage[natbib,style=alphabetic]{biblatex}

% Load the file with bibliography entries
\addbibresource{bibliography.bib}

% Our definitions of macros (see description inside)
\input macros.tex

%%% Title page and various mandatory informational pages
\begin{document}
%%% Title page of the thesis and other mandatory pages

%%% Inscriptions at the opening page of the hard cover

% We usually do not typeset the hard cover, but if you want to do it, change \iffalse to \iftrue
\iffalse

\pagestyle{empty}
\hypersetup{pageanchor=false}
\begin{center}

\large
Charles University

\medskip

Faculty of Mathematics and Physics

\vfill

{\huge\bf\ThesisTypeTitle}

\vfill

{\huge\bf\ThesisTitle\par}

\vfill
\vfill

\hbox to \hsize{\YearSubmitted\hfil \ThesisAuthor}

\end{center}

\newpage\openright
\setcounter{page}{1}

\fi

%%% Title page of the thesis

\pagestyle{empty}
\hypersetup{pageanchor=false}
\begin{center}

\centerline{\mbox{\includegraphics[width=166mm]{img/logo-en.pdf}}}

\vspace{-8mm}
\vfill

{\bf\Large\ThesisTypeTitle}

\vfill

{\LARGE\ThesisAuthor}

\vspace{15mm}

{\LARGE\bfseries\ThesisTitle\par}

\vfill

\Department

\vfill

{
\centerline{\vbox{\halign{\hbox to 0.45\hsize{\hfil #}&\hskip 0.5em\parbox[t]{0.45\hsize}{\raggedright #}\cr
Supervisor of the \ThesisTypeName{} thesis:&\Supervisor \cr
\ifx\ThesisType\TypeRig\else
\noalign{\vspace{2mm}}
Study programme:&\StudyProgramme \cr
\fi
}}}}

\vfill

Prague \YearSubmitted

\end{center}

\newpage

%%% A page with a solemn declaration to the thesis

\openright
\hypersetup{pageanchor=true}
\vglue 0pt plus 1fill

\noindent
I declare that I carried out this \ThesisTypeName{} thesis on my own, and only with the cited
sources, literature and other professional sources.
I understand that my work relates to the rights and obligations under the Act No.~121/2000 Sb.,
the Copyright Act, as amended, in particular the fact that the Charles
University has the right to conclude a license agreement on the use of this
work as a school work pursuant to Section 60 subsection 1 of the Copyright~Act.

\vspace{10mm}

\hbox{\hbox to 0.5\hsize{%
In \hbox to 6em{\dotfill} date \hbox to 6em{\dotfill}
\hss}\hbox to 0.5\hsize{\dotfill\quad}}
\smallskip
\hbox{\hbox to 0.5\hsize{}\hbox to 0.5\hsize{\hfil Author's signature\hfil}}

\vspace{20mm}
\newpage

%%% Dedication

\openright

\noindent
\Dedication

\newpage

%%% Mandatory information page of the thesis

\openright
{\InfoPageFont

\vtop to 0.5\vsize{
\setlength\parindent{0mm}
\setlength\parskip{5mm}

Title:
\ThesisTitle

Author:
\ThesisAuthor

\DeptType:
\Department

Supervisor:
\Supervisor, \SupervisorsDepartment

Abstract:
\Abstract

Keywords:
{\def\sep{\unskip, }\ThesisKeywords}

\vfil
}

% In Czech study programmes, it is mandatory to include Czech meta-data:

\ifx\StudyLanguage\LangCS

\vtop to 0.49\vsize{
\setlength\parindent{0mm}
\setlength\parskip{5mm}

Název práce:
\ThesisTitleCS

Autor:
\ThesisAuthor

\DeptTypeCS:
\DepartmentCS

Vedoucí bakalářské práce:
\Supervisor, \SupervisorsDepartmentCS

Abstrakt:
\AbstractCS

Klíčová slova:
{\def\sep{\unskip, }\ThesisKeywordsCS}

\vfil
}

\fi

}

\newpage

%%% Further pages will be numbered
\pagestyle{plain}


%%% A page with automatically generated table of contents of the thesis

\tableofcontents

%%% Each chapter is kept in a separate file
\chapter*{Introduction}
\addcontentsline{toc}{chapter}{Introduction}
Fluid-structure interactions (FSI) describe situations in which a movable or deformable object interacts with surrounding or internal fluid flow. 
It can have a stable or oscillatory nature, depending on the problem. FSI problems can be found in engineering e.g. aircraft, automobile, spacecraft, or engine design,
and civil engineering e.g. when designing bridges or huge marine structures. FSI problems also arise in biomechanics when modelling blood flow or when studying heart
valve dynamics.

These problems are most of the time too complex to be solved analytically and it is often impossible to conduct relevant experiments. 
If that is the case we need to analyse these problems utilising numerical simulation. Two main approaches for FSI are being used:
\begin{itemize}
    \item Partitioned approach: the governing equations for fluid and solid are solved separately, with two distinct solvers, and coupling is achieved subsequently.
    \item Monolithic approach: the governing equations for both fluid and solid are solved simultaneously, with a single solver.
\end{itemize} 
The partitioned approach often allows us to use existing software which might be advantageous. However, the stability of the coupling algorithm might be an issue. 
Moreover working with a mesh that changes over time is often cumbersome and hard to implement if the geometry of the problem is complex.

We turn our attention to a monolithic approach; specifically the Smoothed Particle Hydrodynamics (SPH) method that was developed in [1] by discretizing the unified formulation of continuum mechanics [2,3].
The governing equations belong to the class of Symmetric Hyperbolic Thermodynamically compatible (SHTC) equations where we substitute the concept of viscosity coefficient by the so-called
particle rearrangement time. The continuum is interpreted as a system of particles connected by bonds and flow is interpreted as bond destruction and rearrangement of particles. 
This is the key idea for a unified description of fluids and solids because (elastic) solids can be interpreted as fluids with infinite relaxation time.

The aim of this paper is to expand on results achieved in paper [1] and test the method for FSI problems. First, we review article [1], the SPH method and we derive the governing equations
from hamiltonian mechanics of continuum. We then proceed to numerical testing namely we test the method for viscoplastic flows. Finally, we put the method in test on the Turek-Hron FSI benchmark.


\chapter{Title of the first chapter}

An~example citation: \cite{Andel07}

\section{Title of the first subchapter of the first chapter}
The thesis should consist of the following steps:
1 Review of paper [1], where an SPH method for both fluids and solids was developed.
2 Review of the SPH method and simple numerical tests using SmoothedParticles.jl framework [2].
3 Review of fluid-structure interaction (FSI) problems in continuum mechanics, for instance the Turek-Hron benchmark [3].
4 An attempt to simulate an FSI problem using SPH.


Abstract:
Smoothed Particle Hydrodynamics (SPH for short) is lagrangian meshfree method that is advantageous 
for simulations of fluids in domain with free boundary. The method in use is based on discretization
of the Symmetric Hyperbolic Thermodynamically Compatible equations (SHTC), that can describe both fluids, elastic solids,
and visco-elasto-plastic solids withing a single framework. Firstly, we review this method. Secondly, we test it on Turek-Hron benchmark and we show that
the method is suitable for simulating FSI (fluid structure interactions) problems and shows remarkable agreement with the data.

Introduction:
Something about SPH
Hamiltonian formulation, Brackets

Smoothed Particle Hydrodynamics is a particle based lagrangian meshfree method for solving partial differential equations. 
It is especially useful when dealing with free surfaces. SPH also conserves energy, linear and angular momentum. 
Among the disadvantages are slower convergence rates and numerical artifacts such as particle clumping or tensile instability [cite].


Fluid-structure interaction (FSI) describes situation where some movable or deformable object interacts with surrounding or internal fluid flow. 
It can have stable or oscillatory nature, depending on the problem. FSI problems can be found in engineering e.g. aircraft, automobile, spacecraft or engine design,
in civil engineering e.g. when designing bridges or huge marine structures. FSI problems also arise in biomechanics when modelling blood flow or when studying heart
valve dynamics.
These problems are often too complex to be solved analyticaly and it is often impossible to conduct relevant experiments. 
If that is the case we need to analyze these problems by means of numerical simulation. Two main aproaches for FSI are being used:
 - Partitioned approach: the governing equations for fluid and for solid are solved separately, with two distinct solvers, coupling is achieved subsequently.
 - Monolithic approach: the governing equations for both fluid and solid are solved simultaneously, with a single solver.
The partitioned approach often allows us to use existing software which might be advantageous. However the stability of the coupling algorithm migh be an issue. 
Moreover working with a mesh that changes over time is often cumbersome and hard to implement if the geometry of the problem is complex.
We turn our attention to monolithic approach specifically the Smoothed Particle Hydrodynamics (SPH) method that was developed in [1] by discretizing the unified formulation of continuum mechanics [2,3].
The governing equations belong to the class of Symmetric Hyperbolic Thermodynamically compatible (SHTC) equations where we subsitute the concept of viscosity coefficient by the so-called
particle rearrangement time. The continuum is interpreted as a system of of particles connected by bonds and flow is interpreted as bond destruction and rearrangement of particles. 
This is the key idea for unified description of fluids and solids because (elastic) solids can be interpreted as fluids with infinite relaxation time.
The aim of this paper is to expand on results achieved in paper [1] and test the method for FSI problems. First, we review article [1], the SPH method and we derive the governing equations
from hamiltonian mechanics of continuum. We then proceed to some examples and we namely test the method for viscoplastic flows. Finally, we put the method in test on the Turek-Hron FSI benchmark.
1. Introduction

The introduction should outline the following elements:

    Background: Describe the significance of fluid-structure interaction (FSI) problems in computational physics and engineering,
    highlighting their relevance across fields such as biomechanics, aerospace, and civil engineering.

    Current Challenges: Discuss the limitations of traditional approaches in simulating both fluids and solids,
    especially in contexts where interactions between the two are critical. Explain how typical FSI models often require
     separate formulations and interactions for fluid and solid domains.

    Objective: State the purpose of your research: developing a unified framework for FSI that leverages the SHTC
    equations for simultaneous treatment of fluid and solid mechanics within a single SPH-based model.

    Contribution: Summarize the unique contribution of this study, specifically the use of the SHTC
    equations discretized with SPH and the demonstration of the model on the Turek-Hron benchmark.



Derivation and the choice of governing equations. Discretization of those equations.
Review of the SPH method and simulation for viscous flow. 
\section{Title of the second subchapter of the first chapter}

\chapter{Title of the second chapter}

\section{SPH}
Smoothed-Particle Hydrodynamics (SPH) is a lagrangian method that works by dividing the continuum into N particles. 
Each particle $a$ has its position $\xx_a$, velocity $\vv_a$, mass $m_a$, density $\rho_a$ etc. Given the lagrangian
structure of SPH it follows that $\dot{\xx}_a = \vv_a$.
Particles interact with each other via the $ smoothing$ $kernel$ function $W_h : \R^d \goto [0, \infty)$. Parameter $h > 0$ is called
$smoothing$ $length$. Kernel function has the following properties:
\begin{itemize}
    \item $W_h \in C^2(\R^d)$
    \item $\int_{\R^d} W_h(\xx) \,d\xx = 1$
    \item $W_h = W_h(|\xx|)$
    \item $h: W_h(r) = \frac{1}{h^d}W_1(r/h)$
    \item $\frac{dW_h}{dr} \leq 0$
\end{itemize}
In this thesis, we will be using Wendland's quintic kernel, which is defined as follows:
\begin{equation}
    W_h(r) =
    \begin{cases}
        \frac{\alpha_d}{h^d} (1 - \frac{r}{2h})^4(1 + \frac{4r}{h})
        & 0 \leq r \leq 2h,\\
        0
        & 2h < r.
    \end{cases} 
\end{equation}
The constant 
\begin{equation}
    \alpha_d =
    \begin{cases}
        \frac{7}{4}\pi &d = 2,\\
        \frac{21}{16}\pi &d = 3
    \end{cases}
\end{equation}
is choosen such that the normalization constraint $\int_{\R^d} W_h(\xx) \,d\xx = 1$ is satisfied.
Functions and partial derivatives are aproximated via the smoothing kernel in two steps. 

Firstly we aproximate $f$ 
respectively ${\pd}_x f$ using convolution:
\begin{equation}
    \begin{aligned}
            f = f * \delta &\approx f * W_h, \\
            {\pd}_x f = {\pd}_x f * \delta &\approx {\pd}_x f * W_h = f * {\pd}_x W_h.    
    \end{aligned}
\end{equation}

Secondly, we aproximate the integral by discreticazing it and summing over all particles in the system:
\begin{equation}
    \begin{aligned}
            (f * W_h)({\xx}_a) = \int f(\mathbf{y}) W_h({\xx}_{a} - \mathbf{y}) d\mathbf{y} \approx \sum_{b}  V_b f_b W_{h, ab}, \\
            (f * {\pd}_{x_i} W_h)({\xx}_a) = \int f(\mathbf{y}) {\pd}_{x_i} W_h({\xx}_a - \mathbf{y}) d\mathbf{y} \approx \sum_{b}  V_b f_b {\pd}_{x_i} W_{h, ab}
    \label{eq:aproxsph}           
    \end{aligned}
\end{equation}
Where $V_b = \rho_b / m_b$ is the volume of the b-th particle, $f_b = f({\xx}_b)$ and $W_{h, ab} = W_h({\xx}_a - {\xx}_b)$.
If there is no room for disambiguation, we will ommit index $h$ and simply write $W_{ab} = W_{h, ab}$. We will use the notation $W'_{ab} = \frac{dW}{dr}(r_{ab})$
and $\nabla W_{ab} = W'_{ab} \frac{{\xx}_{ab}}{r_{ab}}$ where $r_{ab} = |{\xx}_a - {\xx}_b| = |{\xx}_{ab}|$.


\subsection{Discrete density}
The density $\rho_a$ of particle $a$ can be expressed as: 
\begin{equation}
    \rho_a = \rho({\xx}_a) = \sum_{b} m_b W_{ab}.
\end{equation}
It follows that the total differential of $\rho_a$ with respect to the position of the particles $\xx$ is equal to:
\begin{equation}
    d\rho_a = \sum_{b} m_b dW_{ab} = \sum_{b} m_b \sum_{b} \frac{dW_{ab}}{dr} = \sum_{b} \nabla W_{ab} \cdot d{\xx}_{ab}.
\end{equation} 
assuming $r_{ab} > 0$ for each pair of particles $a$ and $b$ and using the fact that for vector $\xx = (x_1,...,x_d)^T$ we have:
\begin{equation}
    dW = \sum_{i=1}^{d} \frac{dW}{dr} \frac{dr}{dx_i} dx_i =  W' \frac{\xx}{r} \cdot d\xx = \nabla W \cdot d\xx
\end{equation}

\subsection{Discrete divergence and gradient}
%We can aproximate divergence $\nabla \cdot \vv$ of vector $\vv = (v_1,..., v_d)^T$ and using ~\Ref{eq:aproxsph} as follows
%\begin{equation}
%    \nabla \cdot \vv = \sum_{i=1}^d \frac{dv_i}{dx_i} = \sum_{i=1}^{d} \sum_{b}  V_b f_b {\pd}_{x_i} W_{h, ab}
%\end{equation}
Assuming all particles have constant mass and differentiating the discrete density $\rho_a$ with respect to time yields
\begin{equation}
    \frac{d\rho_a}{dt} = \sum_b m_b \frac{dW_{ab}}{dt} = \sum_b m_b \nabla W_{ab} \cdot \frac{d{\xx}_{ab}}{dt} = \sum_b m_b {\vv}_{ab} \cdot \nabla W_{ab}
\end{equation}
Comparing the result with continuity equation in the Lagrangian setting 
\begin{equation}
    \frac{d\rho}{dt} = - \rho \nabla \cdot \vv
\end{equation}
motivates us to define the \emph{discrete divergence} operator as follows
\begin{equation}
    (\nabla \cdot \vv)_a = - \frac{1}{\rho_a}\sum_b m_b {\vv}_{ab} \cdot \nabla W_{ab}.
\label{eq:sphdiv}
\end{equation}
Similarly for gradient $\nabla {\vv}$ we have
\begin{equation}
    (\nabla {\vv}_a) = - \frac{1}{\rho_a}\sum_b m_b {\vv}_{ab} \otimes \nabla W_{ab}.
\label{eq:sphgrad}
\end{equation}

\subsection{Discrete velocity gradient}
The velocity gradient $\LL$ satisfies the following equation
\begin{equation}
    \dot{\FF} = \LL \FF
\end{equation}
where $\dot{\FF} = \pd \FF + \vv \cdot \nabla \FF$ is the material derivative of $\FF$ and $\FF = \frac{\pd \xx}{\pd \XX}$ is the deformation gradient.
Using \Ref{eq:sphgrad} we can aproximate $\LL$ as follows
\begin{equation}
    {\LL}_a = \left(\sum_{b}m_b{\vv}_{ab}\otimes \nabla W_{ab}\right) \left(\sum_{b}m_b{\xx}_{ab}\otimes \nabla W_{ab}\right)^{-1}
\end{equation}
\section{Governing equations}
The unified model of continuum for both fluid and solid dynamics reads as follows [cite]:
\begin{subequations}
    
\end{subequations}



\chapter*{Conclusion}
\addcontentsline{toc}{chapter}{Conclusion}


%%% Bibliography
%%% Bibliography (literature used as a source)
%%%
%%% We employ biblatex to construct the bibliography. It processes
%%% citations in the text (e.g., the \cite{...} macro) and looks up
%%% relevant entries in the bibliography.bib file.
%%%
%%% See also biblatex settings in thesis.tex.

%%% Generate the bibliography. Beware that if you cited no works,
%%% the empty list will be omitted completely.

% We let bibliography items stick out of the right margin a little
\def\bibfont{\hfuzz=2pt}

\printbibliography[heading=bibintoc]

%%% If case you prefer to write the bibliography manually (without biblatex),
%%% you can use the following. Please follow the ISO 690 standard and
%%% citation conventions of your field of research.

% \begin{thebibliography}{99}
%
% \bibitem{lamport94}
%   {\sc Lamport,} Leslie.
%   \emph{\LaTeX: A Document Preparation System}.
%   2nd edition.
%   Massachusetts: Addison Wesley, 1994.
%   ISBN 0-201-52983-1.
%
% \end{thebibliography}


%%% Figures used in the thesis (consider if this is needed)
\listoffigures

%%% Tables used in the thesis (consider if this is needed)
%%% In mathematical theses, it could be better to move the list of tables to the beginning of the thesis.
\listoftables

%%% Abbreviations used in the thesis, if any, including their explanation
%%% In mathematical theses, it could be better to move the list of abbreviations to the beginning of the thesis.
\chapwithtoc{List of Abbreviations}

%%% Doctoral theses must contain a list of author's publications
\ifx\ThesisType\TypePhD
\chapwithtoc{List of Publications}
\fi

%%% Attachments to the thesis, if any. Each attachment must be referred to
%%% at least once from the text of the thesis. Attachments are numbered.
%%%
%%% The printed version should preferably contain attachments, which can be
%%% read (additional tables and charts, supplementary text, examples of
%%% program output, etc.). The electronic version is more suited for attachments
%%% which will likely be used in an electronic form rather than read (program
%%% source code, data files, interactive charts, etc.). Electronic attachments
%%% should be uploaded to SIS. Allowed file formats are specified in provision
%%% of the rector no. 72/2017. Exceptions can be approved by faculty's coordinator.
\appendix
\chapter{Attachments}

\section{First Attachment}

\end{document}
