%%% The main file. It contains definitions of basic parameters and includes all other parts.

% Meta-data of your thesis (please edit)
\input metadata.tex

% Generate metadata in XMP format for use by the pdfx package
\input xmp.tex

%% Settings for single-side (simplex) printing
% Margins: left 40mm, right 25mm, top and bottom 25mm
% (but beware, LaTeX adds 1in implicitly)
\documentclass[12pt,a4paper]{report}
\setlength\textwidth{145mm}
\setlength\textheight{247mm}
\setlength\oddsidemargin{15mm}
\setlength\evensidemargin{15mm}
\setlength\topmargin{0mm}
\setlength\headsep{0mm}
\setlength\headheight{0mm}
% \openright makes the following text appear on a right-hand page
\let\openright=\clearpage

%% Settings for two-sided (duplex) printing
% \documentclass[12pt,a4paper,twoside,openright]{report}
% \setlength\textwidth{145mm}
% \setlength\textheight{247mm}
% \setlength\oddsidemargin{14.2mm}
% \setlength\evensidemargin{0mm}
% \setlength\topmargin{0mm}
% \setlength\headsep{0mm}
% \setlength\headheight{0mm}
% \let\openright=\cleardoublepage

%% If the thesis has no printed version, symmetric margins look better
% \documentclass[12pt,a4paper]{report}
% \setlength\textwidth{145mm}
% \setlength\textheight{247mm}
% \setlength\oddsidemargin{10mm}
% \setlength\evensidemargin{10mm}
% \setlength\topmargin{0mm}
% \setlength\headsep{0mm}
% \setlength\headheight{0mm}
% \let\openright=\clearpage

%% Generate PDF/A-2u
\usepackage[a-2u]{pdfx}

%% Prefer Latin Modern fonts
\usepackage{lmodern}
% If we are not using LuaTeX, we need to set up character encoding:
\usepackage{iftex}
\ifpdftex
\usepackage[utf8]{inputenc}
\usepackage[T1]{fontenc}
\usepackage{textcomp}
\fi

%% Further useful packages (included in most LaTeX distributions)
\usepackage{amsmath}        % extensions for typesetting of math
\usepackage{amsfonts}       % math fonts
\usepackage{amsthm}         % theorems, definitions, etc.
\usepackage{bm}             % boldface symbols (\bm)
\usepackage{booktabs}       % improved horizontal lines in tables
\usepackage{caption}        % custom captions of floating objects
\usepackage{dcolumn}        % improved alignment of table columns
\usepackage{floatrow}       % custom float environments
\usepackage{graphicx}       % embedding of pictures
\usepackage{indentfirst}    % indent the first paragraph of a chapter
\usepackage[nopatch=item]{microtype}   % micro-typographic refinement
\usepackage{paralist}       % improved enumerate and itemize
\usepackage[nottoc]{tocbibind} % makes sure that bibliography and the lists
			    % of figures/tables are included in the table
			    % of contents
\usepackage{xcolor}         % typesetting in color

% The hyperref package for clickable links in PDF and also for storing
% metadata to PDF (including the table of contents).
% Most settings are pre-set by the pdfx package.
\hypersetup{unicode}
\hypersetup{breaklinks=true}

% Packages for computer science theses
\usepackage{algpseudocode}  % part of algorithmicx package
\usepackage{algorithm}
\usepackage{fancyvrb}       % improved verbatim environment
\usepackage{listings}       % pretty-printer of source code

% You might want to use cleveref for references
% \usepackage{cleveref}

% Set up formatting of bibliography (references to literature)
% Details can be adjusted in macros.tex.
%
% BEWARE: Different fields of research and different university departments
% have their own customs regarding bibliography. Consult the bibliography
% format with your supervisor.
%
% The basic format according to the ISO 690 standard with numbered references
\usepackage[natbib,style=iso-numeric,sorting=none]{biblatex}
% ISO 690 with alphanumeric references (abbreviations of authors' names)
%\usepackage[natbib,style=iso-alphabetic]{biblatex}
% ISO 690 with references Author (year)
%\usepackage[natbib,style=iso-authoryear]{biblatex}
%
% Some fields of research prefer a simple format with numbered references
% (sorting=none tells that bibliography should be listed in citation order)
%\usepackage[natbib,style=numeric,sorting=none]{biblatex}
% Numbered references, but [1,2,3,4,5] is compressed to [1-5]
%\usepackage[natbib,style=numeric-comp,sorting=none]{biblatex}
% A simple format with alphanumeric references:
%\usepackage[natbib,style=alphabetic]{biblatex}

% Load the file with bibliography entries
\addbibresource{bibliography.bib}

% Our definitions of macros (see description inside)
\input macros.tex

%%% Title page and various mandatory informational pages
\begin{document}
%%% Title page of the thesis and other mandatory pages

%%% Inscriptions at the opening page of the hard cover

% We usually do not typeset the hard cover, but if you want to do it, change \iffalse to \iftrue
\iffalse

\pagestyle{empty}
\hypersetup{pageanchor=false}
\begin{center}

\large
Charles University

\medskip

Faculty of Mathematics and Physics

\vfill

{\huge\bf\ThesisTypeTitle}

\vfill

{\huge\bf\ThesisTitle\par}

\vfill
\vfill

\hbox to \hsize{\YearSubmitted\hfil \ThesisAuthor}

\end{center}

\newpage\openright
\setcounter{page}{1}

\fi

%%% Title page of the thesis

\pagestyle{empty}
\hypersetup{pageanchor=false}
\begin{center}

\centerline{\mbox{\includegraphics[width=166mm]{img/logo-en.pdf}}}

\vspace{-8mm}
\vfill

{\bf\Large\ThesisTypeTitle}

\vfill

{\LARGE\ThesisAuthor}

\vspace{15mm}

{\LARGE\bfseries\ThesisTitle\par}

\vfill

\Department

\vfill

{
\centerline{\vbox{\halign{\hbox to 0.45\hsize{\hfil #}&\hskip 0.5em\parbox[t]{0.45\hsize}{\raggedright #}\cr
Supervisor of the \ThesisTypeName{} thesis:&\Supervisor \cr
\ifx\ThesisType\TypeRig\else
\noalign{\vspace{2mm}}
Study programme:&\StudyProgramme \cr
\fi
}}}}

\vfill

Prague \YearSubmitted

\end{center}

\newpage

%%% A page with a solemn declaration to the thesis

\openright
\hypersetup{pageanchor=true}
\vglue 0pt plus 1fill

\noindent
I declare that I carried out this \ThesisTypeName{} thesis on my own, and only with the cited
sources, literature and other professional sources.
I understand that my work relates to the rights and obligations under the Act No.~121/2000 Sb.,
the Copyright Act, as amended, in particular the fact that the Charles
University has the right to conclude a license agreement on the use of this
work as a school work pursuant to Section 60 subsection 1 of the Copyright~Act.

\vspace{10mm}

\hbox{\hbox to 0.5\hsize{%
In \hbox to 6em{\dotfill} date \hbox to 6em{\dotfill}
\hss}\hbox to 0.5\hsize{\dotfill\quad}}
\smallskip
\hbox{\hbox to 0.5\hsize{}\hbox to 0.5\hsize{\hfil Author's signature\hfil}}

\vspace{20mm}
\newpage

%%% Dedication

\openright

\noindent
\Dedication

\newpage

%%% Mandatory information page of the thesis

\openright
{\InfoPageFont

\vtop to 0.5\vsize{
\setlength\parindent{0mm}
\setlength\parskip{5mm}

Title:
\ThesisTitle

Author:
\ThesisAuthor

\DeptType:
\Department

Supervisor:
\Supervisor, \SupervisorsDepartment

Abstract:
\Abstract

Keywords:
{\def\sep{\unskip, }\ThesisKeywords}

\vfil
}

% In Czech study programmes, it is mandatory to include Czech meta-data:

\ifx\StudyLanguage\LangCS

\vtop to 0.49\vsize{
\setlength\parindent{0mm}
\setlength\parskip{5mm}

Název práce:
\ThesisTitleCS

Autor:
\ThesisAuthor

\DeptTypeCS:
\DepartmentCS

Vedoucí bakalářské práce:
\Supervisor, \SupervisorsDepartmentCS

Abstrakt:
\AbstractCS

Klíčová slova:
{\def\sep{\unskip, }\ThesisKeywordsCS}

\vfil
}

\fi

}

\newpage

%%% Further pages will be numbered
\pagestyle{plain}


%%% A page with automatically generated table of contents of the thesis

\tableofcontents

%%% Each chapter is kept in a separate file
\chapter*{Introduction}
\addcontentsline{toc}{chapter}{Introduction}
Fluid-structure interactions (FSI) describe situations in which a movable or deformable object interacts with surrounding or internal fluid flow. 
It can have a stable or oscillatory nature, depending on the problem. FSI problems can be found in engineering e.g. aircraft, automobile, spacecraft, or engine design,
and civil engineering e.g. when designing bridges or huge marine structures. FSI problems also arise in biomechanics when modelling blood flow or when studying heart
valve dynamics.

These problems are most of the time too complex to be solved analytically and it is often impossible to conduct relevant experiments. 
If that is the case we need to analyse these problems utilising numerical simulation. Two main approaches for FSI are being used:
\begin{itemize}
    \item Partitioned approach: the governing equations for fluid and solid are solved separately, with two distinct solvers, and coupling is achieved subsequently.
    \item Monolithic approach: the governing equations for both fluid and solid are solved simultaneously, with a single solver.
\end{itemize} 
The partitioned approach often allows us to use existing software which might be advantageous. However, the stability of the coupling algorithm might be an issue. 
Moreover working with a mesh that changes over time is often cumbersome and hard to implement if the geometry of the problem is complex.

We turn our attention to a monolithic approach; specifically the Smoothed Particle Hydrodynamics (SPH) method that was developed in [1] by discretizing the unified formulation of continuum mechanics [2,3].
The governing equations belong to the class of Symmetric Hyperbolic Thermodynamically compatible (SHTC) equations where we substitute the concept of viscosity coefficient by the so-called
particle rearrangement time. The continuum is interpreted as a system of particles connected by bonds and flow is interpreted as bond destruction and rearrangement of particles. 
This is the key idea for a unified description of fluids and solids because (elastic) solids can be interpreted as fluids with infinite relaxation time.

The aim of this paper is to expand on results achieved in paper [1] and test the method for FSI problems. First, we review article [1], the SPH method and we derive the governing equations
from hamiltonian mechanics of continuum. We then proceed to numerical testing namely we test the method for viscoplastic flows. Finally, we put the method in test on the Turek-Hron FSI benchmark.


\chapter{Title of the first chapter}

An~example citation: \cite{Andel07}

\section{Title of the first subchapter of the first chapter}
The thesis should consist of the following steps:
1 Review of paper [1], where an SPH method for both fluids and solids was developed.
2 Review of the SPH method and simple numerical tests using SmoothedParticles.jl framework [2].
3 Review of fluid-structure interaction (FSI) problems in continuum mechanics, for instance the Turek-Hron benchmark [3].
4 An attempt to simulate an FSI problem using SPH.


Abstract:
Smoothed Particle Hydrodynamics (SPH for short) is lagrangian meshfree method that is advantageous 
for simulations of fluids in domain with free boundary. The method in use is based on discretization
of the Symmetric Hyperbolic Thermodynamically Compatible equations (SHTC), that can describe both fluids, elastic solids,
and visco-elasto-plastic solids withing a single framework. Firstly, we review this method. Secondly, we test it on Turek-Hron benchmark and we show that
the method is suitable for simulating FSI (fluid structure interactions) problems and shows remarkable agreement with the data.

Introduction:
Something about SPH
Hamiltonian formulation, Brackets

Smoothed Particle Hydrodynamics is a particle based lagrangian meshfree method for solving partial differential equations. 
It is especially useful when dealing with free surfaces. SPH also conserves energy, linear and angular momentum. 
Among the disadvantages are slower convergence rates and numerical artifacts such as particle clumping or tensile instability [cite].


Fluid-structure interaction (FSI) describes situation where some movable or deformable object interacts with surrounding or internal fluid flow. 
It can have stable or oscillatory nature, depending on the problem. FSI problems can be found in engineering e.g. aircraft, automobile, spacecraft or engine design,
in civil engineering e.g. when designing bridges or huge marine structures. FSI problems also arise in biomechanics when modelling blood flow or when studying heart
valve dynamics.
These problems are often too complex to be solved analyticaly and it is often impossible to conduct relevant experiments. 
If that is the case we need to analyze these problems by means of numerical simulation. Two main aproaches for FSI are being used:
 - Partitioned approach: the governing equations for fluid and for solid are solved separately, with two distinct solvers, coupling is achieved subsequently.
 - Monolithic approach: the governing equations for both fluid and solid are solved simultaneously, with a single solver.
The partitioned approach often allows us to use existing software which might be advantageous. However the stability of the coupling algorithm migh be an issue. 
Moreover working with a mesh that changes over time is often cumbersome and hard to implement if the geometry of the problem is complex.
We turn our attention to monolithic approach specifically the Smoothed Particle Hydrodynamics (SPH) method that was developed in [1] by discretizing the unified formulation of continuum mechanics [2,3].
The governing equations belong to the class of Symmetric Hyperbolic Thermodynamically compatible (SHTC) equations where we subsitute the concept of viscosity coefficient by the so-called
particle rearrangement time. The continuum is interpreted as a system of of particles connected by bonds and flow is interpreted as bond destruction and rearrangement of particles. 
This is the key idea for unified description of fluids and solids because (elastic) solids can be interpreted as fluids with infinite relaxation time.
The aim of this paper is to expand on results achieved in paper [1] and test the method for FSI problems. First, we review article [1], the SPH method and we derive the governing equations
from hamiltonian mechanics of continuum. We then proceed to some examples and we namely test the method for viscoplastic flows. Finally, we put the method in test on the Turek-Hron FSI benchmark.
1. Introduction

The introduction should outline the following elements:

    Background: Describe the significance of fluid-structure interaction (FSI) problems in computational physics and engineering,
    highlighting their relevance across fields such as biomechanics, aerospace, and civil engineering.

    Current Challenges: Discuss the limitations of traditional approaches in simulating both fluids and solids,
    especially in contexts where interactions between the two are critical. Explain how typical FSI models often require
     separate formulations and interactions for fluid and solid domains.

    Objective: State the purpose of your research: developing a unified framework for FSI that leverages the SHTC
    equations for simultaneous treatment of fluid and solid mechanics within a single SPH-based model.

    Contribution: Summarize the unique contribution of this study, specifically the use of the SHTC
    equations discretized with SPH and the demonstration of the model on the Turek-Hron benchmark.



Derivation and the choice of governing equations. Discretization of those equations.
Review of the SPH method and simulation for viscous flow. 
\section{Title of the second subchapter of the first chapter}

\chapter{Title of the second chapter}

\section{SPH}
Smoothed-Particle Hydrodynamics (SPH) is a lagrangian method that works by dividing the continuum into N particles. 
Each particle $a$ has its position $\xx_a$, velocity $\vv_a$, mass $m_a$, density $\rho_a$ etc. Given the lagrangian
structure of SPH it follows that $\dot{\xx}_a = \vv_a$.
Particles interact with each other via the $ smoothing$ $kernel$ function $W_h : \R^d \goto [0, \infty)$. Parameter $h > 0$ is called
$smoothing$ $length$. Kernel function has the following properties:
\begin{itemize}
    \item $W_h \in C^2(\R^d)$
    \item $\int_{\R^d} W_h(\xx) \,d\xx = 1$
    \item $W_h = W_h(|\xx|)$
    \item $h: W_h(r) = \frac{1}{h^d}W_1(r/h)$
    \item $\frac{dW_h}{dr} \leq 0$
\end{itemize}
In this thesis, we will be using Wendland's quintic kernel, which is defined as follows:
\begin{equation}
    W_h(r) =
    \begin{cases}
        \frac{\alpha_d}{h^d} (1 - \frac{r}{2h})^4(1 + \frac{4r}{h})
        & 0 \leq r \leq 2h,\\
        0
        & 2h < r.
    \end{cases} 
\end{equation}
The constant 
\begin{equation}
    \alpha_d =
    \begin{cases}
        \frac{7}{4}\pi &d = 2,\\
        \frac{21}{16}\pi &d = 3
    \end{cases}
\end{equation}
is choosen such that the normalization constraint $\int_{\R^d} W_h(\xx) \,d\xx = 1$ is satisfied.
Functions and partial derivatives are aproximated via the smoothing kernel in two steps. 

Firstly we aproximate $f$ 
respectively ${\pd}_x f$ using convolution:
\begin{equation}
    \begin{aligned}
            f = f * \delta &\approx f * W_h, \\
            {\pd}_x f = {\pd}_x f * \delta &\approx {\pd}_x f * W_h = f * {\pd}_x W_h.    
    \end{aligned}
\end{equation}

Secondly, we aproximate the integral by discreticazing it and summing over all particles in the system:
\begin{equation}
    \begin{aligned}
            (f * W_h)({\xx}_a) = \int f(\mathbf{y}) W_h({\xx}_{a} - \mathbf{y}) d\mathbf{y} \approx \sum_{b}  V_b f_b W_{h, ab}, \\
            (f * {\pd}_{x_i} W_h)({\xx}_a) = \int f(\mathbf{y}) {\pd}_{x_i} W_h({\xx}_a - \mathbf{y}) d\mathbf{y} \approx \sum_{b}  V_b f_b {\pd}_{x_i} W_{h, ab}
    \label{eq:aproxsph}           
    \end{aligned}
\end{equation}
Where $V_b = \rho_b / m_b$ is the volume of the b-th particle, $f_b = f({\xx}_b)$ and $W_{h, ab} = W_h({\xx}_a - {\xx}_b)$.
If there is no room for disambiguation, we will ommit index $h$ and simply write $W_{ab} = W_{h, ab}$. We will use the notation $W'_{ab} = \frac{dW}{dr}(r_{ab})$
and $\nabla W_{ab} = W'_{ab} \frac{{\xx}_{ab}}{r_{ab}}$ where $r_{ab} = |{\xx}_a - {\xx}_b| = |{\xx}_{ab}|$.


\subsection{Discrete density}
The density $\rho_a$ of particle $a$ can be expressed as: 
\begin{equation}
    \rho_a = \rho({\xx}_a) = \sum_{b} m_b W_{ab}.
\end{equation}
It follows that the total differential of $\rho_a$ with respect to the position of the particles $\xx$ is equal to:
\begin{equation}
    d\rho_a = \sum_{b} m_b dW_{ab} = \sum_{b} m_b \sum_{b} \frac{dW_{ab}}{dr} = \sum_{b} \nabla W_{ab} \cdot d{\xx}_{ab}.
\end{equation} 
assuming $r_{ab} > 0$ for each pair of particles $a$ and $b$ and using the fact that for vector $\xx = (x_1,...,x_d)^T$ we have:
\begin{equation}
    dW = \sum_{i=1}^{d} \frac{dW}{dr} \frac{dr}{dx_i} dx_i =  W' \frac{\xx}{r} \cdot d\xx = \nabla W \cdot d\xx
\end{equation}

\subsection{Discrete divergence and gradient}
%We can aproximate divergence $\nabla \cdot \vv$ of vector $\vv = (v_1,..., v_d)^T$ and using ~\Ref{eq:aproxsph} as follows
%\begin{equation}
%    \nabla \cdot \vv = \sum_{i=1}^d \frac{dv_i}{dx_i} = \sum_{i=1}^{d} \sum_{b}  V_b f_b {\pd}_{x_i} W_{h, ab}
%\end{equation}
Assuming all particles have constant mass and differentiating the discrete density $\rho_a$ with respect to time yields
\begin{equation}
    \frac{d\rho_a}{dt} = \sum_b m_b \frac{dW_{ab}}{dt} = \sum_b m_b \nabla W_{ab} \cdot \frac{d{\xx}_{ab}}{dt} = \sum_b m_b {\vv}_{ab} \cdot \nabla W_{ab}
\end{equation}
Comparing the result with continuity equation in the Lagrangian setting 
\begin{equation}
    \frac{d\rho}{dt} = - \rho \nabla \cdot \vv
\end{equation}
motivates us to define the \emph{discrete divergence} operator as follows
\begin{equation}
    (\nabla \cdot \vv)_a = - \frac{1}{\rho_a}\sum_b m_b {\vv}_{ab} \cdot \nabla W_{ab}.
\label{eq:sphdiv}
\end{equation}
Similarly for gradient $\nabla {\vv}$ we have
\begin{equation}
    (\nabla {\vv}_a) = - \frac{1}{\rho_a}\sum_b m_b {\vv}_{ab} \otimes \nabla W_{ab}.
\label{eq:sphgrad}
\end{equation}

\subsection{Discrete velocity gradient}
The velocity gradient $\LL$ satisfies the following equation
\begin{equation}
    \dot{\FF} = \LL \FF
\end{equation}
where $\dot{\FF} = \pd \FF + \vv \cdot \nabla \FF$ is the material derivative of $\FF$ and $\FF = \frac{\pd \xx}{\pd \XX}$ is the deformation gradient.
Using \Ref{eq:sphgrad} we can aproximate $\LL$ as follows
\begin{equation}
    {\LL}_a = \left(\sum_{b}m_b{\vv}_{ab}\otimes \nabla W_{ab}\right) \left(\sum_{b}m_b{\xx}_{ab}\otimes \nabla W_{ab}\right)^{-1}
\end{equation}
\section{Governing equations}
The unified model of continuum for both fluid and solid dynamics reads as follows [cite]:
\begin{subequations}
    
\end{subequations}


%%%% Fiktivní kapitola s ukázkami sazby

\chapter{Nápověda k~sazbě}

\section{Úprava práce}

Vlastní text práce je uspořádaný hierarchicky do kapitol a podkapitol,
každá kapitola začíná na nové straně. Text je zarovnán do bloku. Nový odstavec
se obvykle odděluje malou vertikální mezerou a odsazením prvního řádku. Grafická
úprava má být v~celém textu jednotná.

Práce se tiskne na bílý papír formátu A4. Okraje musí ponechat dost místa na vazbu:
doporučen je horní, dolní a pravý okraj $25\,\rm mm$, levý okraj $40\,\rm mm$.
Číslují se všechny strany kromě obálky a informačních stran na začátku práce;
první číslovaná strana bývá obvykle ta s~obsahem.

Písmo se doporučuje dvanáctibodové ($12\,\rm pt$) se standardní vzdáleností mezi řádky
(pokud píšete ve Wordu nebo podobném programu, odpovídá tomu řádkování $1,5$; v~\TeX{}u
není potřeba nic přepínat).

Primárně je doporučován jednostranný tisk (příliš tenkou práci lze obtížně svázat).
Delší práce je lepší tisknout oboustranně a přizpůsobit tomu velikosti okrajů:
$40\,\rm mm$ má vždy \emph{vnitřní} okraj. Rub titulního listu zůstává nepotištěný.

Zkratky použité v textu musí být vysvětleny vždy u prvního výskytu zkratky (v~závorce nebo
v poznámce pod čarou, jde-li o složitější vysvětlení pojmu či zkratky). Pokud je zkratek
více, připojuje se seznam použitých zkratek, včetně jejich vysvětlení a/nebo odkazů
na definici.

Delší převzatý text jiného autora je nutné vymezit uvozovkami nebo jinak vyznačit a řádně
citovat.

\section{Jednoduché příklady}

K~různým účelům se hodí různé typy písma.
Pro běžný text používáme vzpřímené patkové písmo.
Chceme-li nějaký pojem zvýraznit (třeba v~okamžiku definice), používáme obvykle
\textit{kurzívu} nebo \textbf{tučné písmo.}
Text matematických vět se obvykle tiskne pro zdůraznění \textsl{skloněným (slanted)} písmem;
není-li k~dispozici, může být zastoupeno \textit{kurzívou.}
Text, který je chápan doslova (například ukázky programů) píšeme \texttt{psacím strojem}.
Důležité je být ve volbě písma konzistentní napříč celou prací.

Čísla v~českém textu obvykle sázíme v~matematickém režimu s~desetinnou čárkou:
%%% Bez \usepackage{icomma}:
% $\pi \doteq 3{,}141\,592\,653\,589$.
%%% S \usepackage{icomma}:
$\pi \doteq 3,141\,592\,653\,589$.
V~matematických textech je často lepší používat desetinnou tečku
(pro lepší odlišení od čárky v~roli oddělovače).
Nestřídejte však obojí.
Numerické výsledky se uvádějí s~přiměřeným počtem desetinných míst.

Mezi číslo a jednotku patří úzká mezera: šířka stránky A4 činí $210\,\rm mm$, což si
pamatuje pouze $5\,\%$ autorů. Pokud ale údaj slouží jako přívlastek, mezeru vynecháváme:
$25\rm mm$ okraj, $95\%$ interval spolehlivosti.

Rozlišujeme různé druhy pomlček:
červeno-černý (krátká pomlčka),
strana 16--22 (střední),
$45-44$ (matematické minus),
a~toto je --- jak se asi dalo čekat --- vložená věta ohraničená dlouhými pomlčkami.

V~českém textu se používají \uv{české} uvozovky, nikoliv ``anglické''.

% V tomto odstavci se vlnka zviditelňuje
{
\def~{{\tt\char126}}
Na některých místech je potřeba zabránit lámání řádku (v~\TeX{}u značíme vlnovkou):
u~předložek (neslabičnych, nebo obecně jednopísmenných), vrchol~$v$, před $k$~kroky,
a~proto, \dots{} obecně kdekoliv, kde by při rozlomení čtenář \uv{ško\-brt\-nul}.
}

\section{Matematické vzorce a výrazy}

Proměnné sázíme kurzívou (to \TeX{} v~matematickém módu dělá sám, ale
nezapomínejte na to v~okolním textu a také si matematický mód zapněte).
Názvy funkcí sázíme vzpřímeně. Tedy například:
$\var(X) = \E X^2 - \bigl(\E X \bigr)^2$.

Zlomky uvnitř odstavce (třeba $\frac{5}{7}$ nebo $\frac{x+y}{2}$) mohou
být příliš stísněné, takže je lepší sázet jednoduché zlomky s~lomítkem:
$5/7$, $(x+y)/2$.

Není předepsáno, jakým písmem označovat jednotlivé druhy matematických objektů
(matice, vektory, atd.), ale značení pro tentýž druh objektu musí být v~celé
práci používáno stejně. Podobně používáte-li více různých typů závorek, je třeba
dělat to v~celé práci konzistentně.

Nechť
\[   % LaTeXová náhrada klasického TeXového $$
\mathbf{X} = \begin{pmatrix}
      \T{\bm x_1} \\
      \vdots \\
      \T{\bm x_n}
      \end{pmatrix}.
\]
Povšimněme si tečky za~maticí. Byť je matematický text vysázen
ve~specifickém prostředí, stále je gramaticky součástí věty a~tudíž je
zapotřebí neopomenout patřičná interpunkční znaménka. Obecně nechceme
sázet vzorce jeden za druhým a raději je propojíme textem.

Výrazy, na které chceme později odkazovat, je vhodné očíslovat:
\begin{equation}\label{eq01:Xmat}
\mathbf{X} = \begin{pmatrix}
      \T{\bm x_1} \\
      \vdots \\
      \T{\bm x_n}
      \end{pmatrix}.
\end{equation}
Výraz \eqref{eq01:Xmat} definuje matici $\mathbf{X}$. Pro lepší čitelnost
a~přehlednost textu je vhodné číslovat pouze ty výrazy, na které se
autor někde v~další části textu odkazuje. To jest, nečíslujte
automaticky všechny výrazy vysázené některým z~matematických
prostředí.

Zarovnání vzorců do několika sloupečků:
\begin{alignat*}{3}
S(t) &= \pr(T > t),    &\qquad t&>0       &\qquad&\text{ (zprava spojitá),}\\
F(t) &= \pr(T \leq t), &\qquad t&>0       &\qquad&\text{ (zprava spojitá).}
\end{alignat*}

Dva vzorce se spojovníkem:
\begin{equation}\label{eq01:FS}
\left.
\begin{aligned}
S(t) &= \pr(T > t) \\[1ex]
F(t) &= \pr(T \leq t)
\end{aligned}
\;	% zde pomůže ručně vynechat trochu místa
\right\}
\quad t>0 \qquad \text{(zprava spojité).}
\end{equation}

Dva centrované nečíslované vzorce:
\begin{gather*}
\bm Y = \mathbf{X}\bm\beta + \bm\varepsilon, \\[1ex]
\mathbf{X} = \begin{pmatrix} 1 & \T{\bm x_1} \\ \vdots & \vdots \\ 1 &
  \T{\bm x_n} \end{pmatrix}.
\end{gather*}
Dva centrované číslované vzorce:
\begin{gather}
\bm Y = \mathbf{X}\bm\beta + \bm\varepsilon, \label{eq02:Y}\\[1ex]
\mathbf{X} = \begin{pmatrix} 1 & \T{\bm x_1} \label{eq03:X}\\ \vdots & \vdots \\ 1 &
  \T{\bm x_n} \end{pmatrix}.
\end{gather}

Definice rozdělená na dva případy:
\[
P_{r-j}=
\begin{cases}
0, & \text{je-li $r-j$ liché},\\
r!\,(-1)^{(r-j)/2}, & \text{je-li $r-j$ sudé}.
\end{cases}
\]
Všimněte si použití interpunkce v této konstrukci. Čárky a tečky se
dávají na místa, kam podle jazykových pravidel patří.

\begin{align}
x& = y_1-y_2+y_3-y_5+y_8-\dots = && \text{z \eqref{eq02:Y}} \nonumber\\
& = y'\circ y^* = && \text{podle \eqref{eq03:X}} \nonumber\\
& = y(0) y' && \text {z Axiomu 1.}
\end{align}


Dva zarovnané vzorce nečíslované (povšimněte si větších závorek, aby se do nich
vešel vyšší vzorec):
\begin{align*}
L(\bm\theta) &= \prod_{i=1}^n f_i(y_i;\,\bm\theta), \\
\ell(\bm\theta) &= \log\bigl\{L(\bm\theta)\bigr\} =
\sum_{i=1}^n \log\bigl\{f_i(y_i;\,\bm\theta)\bigr\}.
\end{align*}
Dva zarovnané vzorce, první číslovaný:
\begin{align}
L(\bm\theta) &= \prod_{i=1}^n f_i(y_i;\,\bm\theta), \label{eq01:L} \\
\ell(\bm\theta) &= \log\bigl\{L(\bm\theta)\bigr\} =
\sum_{i=1}^n \log\bigl\{f_i(y_i;\,\bm\theta)\bigr\}. \nonumber
\end{align}

Vzorec na dva řádky, první řádek zarovnaný vlevo, druhý vpravo, nečíslovaný:
\begin{multline*}
\ell(\mu,\,\sigma^2) = \log\bigl\{L(\mu,\,\sigma^2)\bigr\} =
\sum_{i=1}^n \log\bigl\{f_i(y_i;\,\mu,\,\sigma^2)\bigr\}= \\
  = -\,\frac{n}{2}\,\log(2\pi\sigma^2) \,-\,
\frac{1}{2\sigma^2}\sum_{i=1}^n\,(y_i - \mu)^2.
\end{multline*}

Vzorec na dva řádky, zarovnaný na $=$, číslovaný uprostřed:
\begin{equation}\label{eq01:ell}
\begin{split}
\ell(\mu,\,\sigma^2) &= \log\bigl\{L(\mu,\,\sigma^2)\bigr\} =
\sum_{i=1}^n \log\bigl\{f(y_i;\,\mu,\,\sigma^2)\bigr\}= \\
& = -\,\frac{n}{2}\,\log(2\pi\sigma^2) \,-\,
\frac{1}{2\sigma^2}\sum_{i=1}^n\,(y_i - \mu)^2.
\end{split}
\end{equation}

\section{Definice, věty, důkazy, \dots}

Konstrukce typu definice, věta, důkaz, příklad, \dots je vhodné
odlišit od okolního textu a~případně též číslovat s~možností použití
křížových odkazů. Pro každý typ těchto konstrukcí je vhodné mít
v~souboru s~makry (\texttt{makra.tex}) nadefinované jedno prostředí,
které zajistí jak vizuální odlišení od okolního textu, tak
automatické číslování s~možností křížově odkazovat.

\begin{definice}\label{def01:1}
  Nechť náhodné veličiny $X_1,\dots,X_n$ jsou definovány na témž
  prav\-dě\-po\-dob\-nost\-ním prostoru $(\Omega,\,\mathcal{A},\,\pr)$. Pak
  vektor $\bm X = \T{(X_1,\dots,X_n)}$ nazveme \emph{náhodným
    vektorem}.
\end{definice}

\begin{definice}[náhodný vektor]\label{def01:2}
  Nechť náhodné veličiny $X_1,\dots,X_n$ jsou definovány na témž
  pravděpodobnostním prostoru $(\Omega,\,\mathcal{A},\,\pr)$. Pak
  vektor $\bm X = \T{(X_1,\dots,X_n)}$ nazveme \emph{náhodným
    vektorem}.
\end{definice}
Definice~\ref{def01:1} ukazuje použití prostředí pro sazbu definice
bez titulku, definice~\ref{def01:2} ukazuje použití prostředí pro
sazbu definice s~titulkem.

\begin{veta}\label{veta01:1}
  Náhodný vektor $\bm X$ je měřitelné zobrazení prostoru
  $(\Omega,\,\mathcal{A},\,\pr)$ do $(\R_n,\,\mathcal{B}_n)$.
\end{veta}

\begin{lemma}[\citet{Andel07}, str. 29]\label{veta01:2}
  Náhodný vektor $\bm X$ je měřitelné zobrazení prostoru
  $(\Omega,\,\mathcal{A},\,\pr)$ do $(\R_n,\,\mathcal{B}_n)$.
\end{lemma}
\begin{dukaz}
  Jednotlivé kroky důkazu jsou podrobně popsány v~práci \citet[str.
  29]{Andel07}.
\end{dukaz}
Věta~\ref{veta01:1} ukazuje použití prostředí pro sazbu matematické
věty bez titulku, lemma~\ref{veta01:2} ukazuje použití prostředí pro
sazbu matematické věty s~titulkem. Lemmata byla zavedena v~hlavním
souboru tak, že sdílejí číslování s~větami.

%%%% Fiktivní kapitola s ukázkami citací

\chapter{Odkazy na literaturu}

Při zpracování bibliografie (přehledu použitých zdrojů) se řídíme
normou ISO 690 a zvyklostmi oboru. V~\LaTeX{}u nám pomohou balíčky
\textsf{biblatex}, \textsf{biblatex-iso690}.
Zdroje definujeme v~souboru \texttt{literatura.bib} a pak se na ně v~textu
práce odkazujeme pomocí makra \verb|\cite|. Tím vznikne odkaz v~textu
a~odpovídající položka v~seznamu literatury.

V~matematickém textu obvykle odkazy sázíme ve tvaru
\uv{Jméno autora/autorů [číslo odkazu]}, případně
\uv{Jméno autora/autorů (rok vydání)}.
V~českém/slovenském textu je potřeba se navíc vypořádat
s~nutností skloňovat jméno autora, respektive přechylovat jméno
autorky.
K~doplňování jmen se hodí příkazy \verb|\citet|, \verb|\citep|
z~balíčku \textsf{natbib}, ale je třeba mít na paměti, že
produkují referenci se jménem autora/autorů v~prvním pádě a~jména
autorek jsou nepřechýlena.

Jména časopisů lze uvádět zkráceně, ale pouze v~kodifikované podobě.

Při citování je třeba se vyhnout neověřitelným, nedohledatelným a nestálým zdrojům.
Doporučuje se pokud možno necitovat osobní sdělení, náhodně nalezené webové stránky,
poznámky k přednáškám apod. Citování spolehlivých elektronických zdrojů (maji ISSN
nebo DOI) a webových stránek oficiálních instituci je zcela v pořádku. Citujeme-li
elektronické zdroje, je třeba uvést URL, na němž se zdroj nachází, a~datum přístupu
ke zdroji.

\section{Několik ukázek}

Aktuální verzi této šablony najdete v~gitovém repozitáři \cite{ThesisTemplate}.
Také se může hodit prohlédnout si další návody udržované Martinem Marešem
\cite{ThesisWeb}.

Mezi nejvíce citované statistické články patří práce Kaplana a~Meiera a~Coxe
\cite{KaplanMeier58, Cox72}. \citet{Student08} napsal článek o~t-testu.

Prof. Anděl je autorem učebnice matematické statistiky \cite{Andel98}.
Teorii odhadu se věnuje práce \citet{LehmannCasella98}.
V~případě odkazů na specifickou informaci
(definice, důkaz, \dots) uvedenou v~knize bývá užitečné uvést
specificky číslo kapitoly, číslo věty atp. obsahující požadovanou
informaci, např. viz \citet[Věta 4.22]{Andel07}.

Mnoho článků je výsledkem spolupráce celé řady osob. Při odkazování
v~textu na článek se třemi autory obvykle při prvním výskytu uvedeme
plný seznam: \citet*{DempsterLairdRubin77} představili koncept EM
algoritmu. Respektive: Koncept EM algoritmu byl představen v~práci
Dempstera, Lairdové a~Rubina~\cite{DempsterLairdRubin77}. Při každém
dalším výskytu již používáme zkrácenou verzi:
\citet{DempsterLairdRubin77} nabízejí též několik příkladů použití EM
algoritmu. Respektive: Několik příkladů použití EM algoritmu lze
nalézt též v~práci Dempstera a~kol.~\cite{DempsterLairdRubin77}.

U~článku s~více než třemi autory odkazujeme vždy zkrácenou formou:
První výsledky projektu ACCEPT jsou uvedeny v~práci Genbergové a~kol.~\cite{Genberget08}.
V~textu \emph{nenapíšeme:} První výsledky
projektu ACCEPT jsou uvedeny v~práci \citet*{Genberget08}.

%%%% Fiktivní kapitola s ukázkami tabulek, obrázků a kódu

\chapter{Tabulky, obrázky, programy}

Používání tabulek a grafů v~odborném textu má některá společná
pravidla a~některá specifická. Tabulky a grafy neuvádíme přímo do
textu, ale umístíme je buď na samostatné stránky nebo na vyhrazené
místo v~horní nebo dolní části běžných stránek. \LaTeX\ se o~umístění
plovoucích grafů a tabulek postará automaticky.

Každý graf a tabulku
očíslujeme a umístíme pod ně legendu. Legenda má popisovat obsah grafu
či tabulky tak podrobně, aby jim čtenář rozuměl bez důkladného
studování textu práce.

Na každou tabulku a graf musí být v~textu odkaz
pomocí jejich čísla. Na příslušném místě textu pak shrneme ty
nejdůležitější závěry, které lze z~tabulky či grafu učinit. Text by
měl být čitelný a srozumitelný i~bez prohlížení tabulek a grafů a
tabulky a grafy by měly být srozumitelné i~bez podrobné četby textu.

Na tabulky a grafy odkazujeme pokud možno nepřímo v~průběhu běžného
toku textu; místo \emph{\uv{Tabulka~\ref{tab03:Nejaka} ukazuje, že
    muži jsou v~průměru o~$9,9\,\rm kg$ těžší než ženy}} raději napíšeme
\emph{\uv{Muži jsou o~$9,9\,\rm kg$ těžší než ženy (viz
    Tabulka~\ref{tab03:Nejaka})}}.

\section{Tabulky}

\begin{table}[b!]

\centering
%%% Tabulka používá následující balíčky:
%%%   - booktabs (\toprule, \midrule, \bottomrule)
%%%   - dcolumn (typ sloupce D: vycentrovaná čísla zarovnaná na
%%%     desetinnou čárku
%%%     Všimněte si, že ve zdrojovém kódu jsou desetinné tečky, ale
%%%     tisknou se čárky.
%%% Dále používáme příkazy \pulrad a \mc definované v makra.tex

\begin{tabular}{l@{\hspace{1.5cm}}D{.}{,}{3.2}D{.}{,}{1.2}D{.}{,}{2.3}}
\toprule
 & \mc{} & \mc{\textbf{Směrod.}} & \mc{} \\
\pulrad{\textbf{Efekt}} & \mc{\pulrad{\textbf{Odhad}}} & \mc{\textbf{chyba}$^a$} &
\mc{\pulrad{\textbf{P-hodnota}}} \\
\midrule
Abs. člen     & -10.01 & 1.01 & \mc{---} \\
Pohlaví (muž) & 9.89   & 5.98 & 0.098 \\
Výška (cm)    & 0.78   & 0.12 & <0.001 \\
\bottomrule
\multicolumn{4}{l}{\footnotesize \textit{Pozn:}
$^a$ Směrodatná chyba odhadu metodou Monte Carlo.}
\end{tabular}

\caption{Maximálně věrohodné odhady v~modelu M.}\label{tab03:Nejaka}

\end{table}

U~\textbf{tabulek} se doporučuje dodržovat následující pravidla:

\begin{itemize} %% nebo compactitem z balíku paralist
\item Vyhýbat se svislým linkám. Silnějšími vodorovnými linkami
  oddělit tabulku od okolního textu včetně legendy, slabšími
  vodorovnými linkami oddělovat záhlaví sloupců od těla tabulky a
  jednotlivé části tabulky mezi sebou. V~\LaTeX u tuto podobu tabulek
  implementuje balík \texttt{booktabs}. Chceme-li výrazněji oddělit
  některé sloupce od jiných, vložíme mezi ně větší mezeru.
\item Neměnit typ, formát a význam obsahu políček v~tomtéž sloupci
  (není dobré do téhož sloupce zapisovat tu průměr, onde procenta).
\item Neopakovat tentýž obsah políček mnohokrát za sebou. Máme-li
  sloupec \textit{Rozptyl}, který v~prvních deseti řádcích obsahuje
  hodnotu $0,5$ a v~druhých deseti řádcích hodnotu $1,5$, pak tento
  sloupec raději zrušíme a vyřešíme to jinak. Například můžeme tabulku
  rozdělit na dvě nebo do ní vložit popisné řádky, které informují
o~nějaké proměnné hodnotě opakující se v~následujícím oddíle tabulky
  (např. \emph{\uv{Rozptyl${}=0,5$}} a níže \emph{\uv{Rozptyl${}=
      1,5$}}).
\item Čísla v~tabulce zarovnávat na desetinnou čárku.
\item V~tabulce je někdy potřebné používat zkratky, které se jinde
nevyskytují. Tyto zkratky můžeme vysvětlit v~legendě nebo
v~poznámkách pod tabulkou. Poznámky pod tabulkou můžeme využít i
k~podrobnějšímu vysvětlení významu  některých sloupců nebo hodnot.
\end{itemize}

\section{Obrázky}

Dodejme ještě několik rad týkajících se obrázků a grafů.

\begin{itemize}
\item Graf by měl být vytvořen ve velikosti, v~níž bude použit
  v~práci. Zmenšení příliš velkého grafu vede ke špatné čitelnosti
  popisků.
\item Osy grafu musí být řádně popsány ve stejném jazyce, v~jakém je
  psána práce (absenci diakritiky lze tolerovat). Kreslíme-li graf
  hmotnosti proti výšce, nenecháme na nich popisky \texttt{ht} a
  \texttt{wt}, ale osy popíšeme \emph{Výška [cm]} a~\emph{Hmotnost
    [kg]}. Kreslíme-li graf funkce $h(x)$, popíšeme osy $x$ a $h(x)$.
  Každá osa musí mít jasně určenou škálu.
\item Chceme-li na dvourozměrném grafu vyznačit velké množství bodů,
  dáme pozor, aby se neslily do jednolité černé tmy. Je-li bodů mnoho,
  zmenšíme velikost symbolu, kterým je vykreslujeme, anebo vybereme
  jen malou část bodů, kterou do grafu zaneseme. Grafy, které obsahují
  tisíce bodů, dělají problémy hlavně v~elektronických dokumentech,
  protože výrazně zvětšují velikost souborů.
\item Budeme-li práci tisknout černobíle, vyhneme se používání barev.
  Čáry roz\-li\-šu\-je\-me typem (plná, tečkovaná, čerchovaná,\ldots), plochy
  dostatečně roz\-díl\-ný\-mi intensitami šedé nebo šrafováním. Význam
  jednotlivých typů čar a~ploch vysvětlíme buď v~textové legendě ke
  grafu anebo v~grafické legendě, která je přímo součástí obrázku.
\item Vyhýbejte se bitmapovým obrázkům o~nízkém rozlišení a zejména
  JPEGům (zuby a kompresní artefakty nevypadají na papíře pěkně).
  Lepší je vytvářet obrázky vektorově a vložit do textu jako PDF.
\end{itemize}

\section{Programy}

Algoritmy, výpisy programů a popis interakce s~programy je vhodné
odlišit od ostatního textu. Pro programy se hodí prostředí \texttt{lstlisting}
z~\LaTeX{}ového balíčku \texttt{listings}, které umí i syntakticky zvýrazňovat
běžné programovací jazyky. Většinou ho chceme obalit prostředím \texttt{listing},
čímž z~něj uděláme plovoucí objekt s~popiskem (viz Program~\ref{helloworld}).

Pro algoritmy zapsané v~pseudokódu můžeme použít prostředí \texttt{algorithmic}
z~balíčku \texttt{algpseudocode}. Plovoucí objekt z~něj uděláme obalením prostředím
\texttt{algorithm}. Příklad najdete v~Algoritmu~\ref{isprime}.

\begin{listing}
\begin{lstlisting}
#include <stdio.h>

int main(void)
{
	printf("Hello, world!\n");
	return 0;
}
\end{lstlisting}
\caption{Můj první program}
\label{helloworld}
\end{listing}

\begin{algorithm}
\begin{algorithmic}[1]  % [1] způsobí, že číslujeme kroky algoritmu
\Function{IsPrime}{$x$}
	\State $r \gets \mbox{rovnoměrně náhodné celé číslo mezi 2 a~$x-1$}$
	\State $z \gets x \bmod r$
	\If{$z>0$}
		\State Vrátíme \textsc{ne} \Comment{Našli jsme dělitele}
	\Else
		\State Vrátíme \textsc{ano} \Comment{Možná se mýlíme}
	\EndIf
\EndFunction
\end{algorithmic}
\caption{Primitivní pravděpodobnostní test prvočíselnosti. Pokud odpoví \textsc{ne},
	číslo~$x$ určitě není prvočíslem. Pokud odpoví \textsc{ano}, nejspíš se mýlí.}
\label{isprime}
\end{algorithm}

Další možností je použití {\LaTeX}o\-vé\-ho balíčku
\texttt{fancyvrb} (fancy verbatim), pomocí něhož je v~souboru \texttt{makra.tex}
nadefinováno prostředí \texttt{code}. Pomocí něho lze vytvořit
např. následující ukázky.

V~základním nastavení dostaneme:

\begin{code}
> mean(x)
[1] 158.90
> objekt$prumer
[1] 158.90
\end{code}
%$
Můžeme si říci o~menší písmo:
\begin{code}[fontsize=\footnotesize]
> mean(x)
[1] 158.90
> objekt$prumer
[1] 158.90
\end{code}
%$
Nebo vypnout rámeček:
\begin{code}[frame=none]
> mean(x)
[1] 158.90
> objekt$prumer
[1] 158.90
\end{code}
%$
Případně si říci o~užší rámeček:
\begin{code}[xrightmargin=20em]
> mean(x)
[1] 158.90
> objekt$prumer
[1] 158.90
\end{code}
%$

\begin{figure}[p]\centering
\includegraphics[width=140mm, height=140mm]{img/ukazka-obr01}
% Příponu není potřeba explicitně uvádět, pdflatex automaticky hledá pdf.
% Rozměry také není nutné uvádět.
\caption{Náhodný výběr z~rozdělení $\mathcal{N}_2(\boldsymbol{0},\,I)$.}
\label{obr03:Nvyber}

\end{figure}

\begin{figure}[p]\centering
\includegraphics[width=140mm, height=140mm]{img/ukazka-obr02}
\caption{Hustoty několika normálních rozdělení.}
\label{obr03:Nhust}
\end{figure}

\begin{figure}[p]\centering
\includegraphics[width=140mm, height=198mm]{img/ukazka-obr03}
\caption{Hustoty několika normálních rozdělení.}
\label{obr03:Nhust:podruhe}

\end{figure}

%%%% Fiktivní kapitola s instrukcemi k PDF/A

\chapter{Formát PDF/A}

Opatření rektora č. 13/2017 určuje, že elektronická podoba závěrečných
prací musí být odevzdávána ve formátu PDF/A úrovně 1a nebo 2u. To jsou
profily formátu PDF určující, jaké vlastnosti PDF je povoleno používat,
aby byly dokumenty vhodné k~dlouhodobé archivaci a dalšímu automatickému
zpracování. Dále se budeme zabývat úrovní 2u, kterou sázíme \TeX{}em.

Mezi nejdůležitější požadavky PDF/A-2u patří:

\begin{itemize}

\item Všechny fonty musí být zabudovány uvnitř dokumentu. Nejsou přípustné
odkazy na externí fonty (ani na \uv{systémové}, jako je Helvetica nebo Times).

\item Fonty musí obsahovat tabulku ToUnicode, která definuje převod z~kódování
znaků použitého uvnitř fontu to Unicode. Díky tomu je možné z~dokumentu
spolehlivě extrahovat text.

\item Dokument musí obsahovat metadata ve formátu XMP a je-li barevný,
pak také formální specifikaci barevného prostoru.

\end{itemize}

Tato šablona používá balíček {\tt pdfx,} který umí \LaTeX{} nastavit tak,
aby požadavky PDF/A splňoval. Metadata v~XMP se generují automaticky podle
informací v~souboru {\tt prace.xmpdata} (na vygenerovaný soubor se můžete
podívat v~{\tt pdfa.xmpi}).

Validitu PDF/A můžete zkontrolovat pomocí nástroje VeraPDF, který je
k~dispozici na \url{http://verapdf.org/}.

Pokud soubor nebude validní, mezi obvyklé příčiny patří používání méně
obvyklých fontů (které se vkládají pouze v~bitmapové podobě a/nebo bez
unicodových tabulek) a vkládání obrázků v~PDF, které samy o~sobě standard
PDF/A nesplňují.

Další postřehy o~práci s~PDF/A najdete na \url{http://mj.ucw.cz/vyuka/bc/pdfaq.html}.

\chapter*{Conclusion}
\addcontentsline{toc}{chapter}{Conclusion}


%%% Bibliography
%%% Bibliography (literature used as a source)
%%%
%%% We employ biblatex to construct the bibliography. It processes
%%% citations in the text (e.g., the \cite{...} macro) and looks up
%%% relevant entries in the bibliography.bib file.
%%%
%%% See also biblatex settings in thesis.tex.

%%% Generate the bibliography. Beware that if you cited no works,
%%% the empty list will be omitted completely.

% We let bibliography items stick out of the right margin a little
\def\bibfont{\hfuzz=2pt}

\printbibliography[heading=bibintoc]

%%% If case you prefer to write the bibliography manually (without biblatex),
%%% you can use the following. Please follow the ISO 690 standard and
%%% citation conventions of your field of research.

% \begin{thebibliography}{99}
%
% \bibitem{lamport94}
%   {\sc Lamport,} Leslie.
%   \emph{\LaTeX: A Document Preparation System}.
%   2nd edition.
%   Massachusetts: Addison Wesley, 1994.
%   ISBN 0-201-52983-1.
%
% \end{thebibliography}


%%% Figures used in the thesis (consider if this is needed)
\listoffigures

%%% Tables used in the thesis (consider if this is needed)
%%% In mathematical theses, it could be better to move the list of tables to the beginning of the thesis.
\listoftables

%%% Abbreviations used in the thesis, if any, including their explanation
%%% In mathematical theses, it could be better to move the list of abbreviations to the beginning of the thesis.
\chapwithtoc{List of Abbreviations}

%%% Doctoral theses must contain a list of author's publications
\ifx\ThesisType\TypePhD
\chapwithtoc{List of Publications}
\fi

%%% Attachments to the thesis, if any. Each attachment must be referred to
%%% at least once from the text of the thesis. Attachments are numbered.
%%%
%%% The printed version should preferably contain attachments, which can be
%%% read (additional tables and charts, supplementary text, examples of
%%% program output, etc.). The electronic version is more suited for attachments
%%% which will likely be used in an electronic form rather than read (program
%%% source code, data files, interactive charts, etc.). Electronic attachments
%%% should be uploaded to SIS. Allowed file formats are specified in provision
%%% of the rector no. 72/2017. Exceptions can be approved by faculty's coordinator.
\appendix
\chapter{Attachments}

\section{First Attachment}

\end{document}
