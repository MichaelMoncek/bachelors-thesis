%%% Please fill in basic information on your thesis, which will be automatically
%%% inserted at the right places. You need to replace \xxx{...} by real data.

% Type of your thesis:
%	"bc" for Bachelor's
%	"mgr" for Master's
%	"phd" for PhD
%	"rig" for rigorosum
\def\ThesisType{bc}

% Language of your study programme:
%	"cs" for Czech
%	"en" for English
\def\StudyLanguage{cs}

% Thesis title in English (exactly as in the official assignment)
% (Note: \xxx is a "ToDo label" which makes the unfilled visible. Remove it.)
\def\ThesisTitle{\xxx{Modelling Fluid-Structure Interaction using Smoothed Particle Hydrodynamics}}

% Author of the thesis (you)
\def\ThesisAuthor{\xxx{Michael Monček}}

% Year when the thesis is submitted
\def\YearSubmitted{\xxx{2025}}

% Name of the department or institute, where the work was officially assigned
% (according to the Organizational Structure of MFF UK in English,
% see https://www.mff.cuni.cz/en/faculty/organizational-structure,
% or a full name of a department outside MFF)
\def\Department{\xxx{Mathematical Institute of Charles University}}

% Is it a department (katedra), or an institute (ústav)?
\def\DeptType{\xxx{Institute}}

% Thesis supervisor: name, surname and titles
\def\Supervisor{\xxx{Michal Pavelka}}

% Supervisor's department (again according to Organizational structure of MFF)
\def\SupervisorsDepartment{\xxx{Mathematical Institute of Charles University}}

% Study programme (does not apply to rigorosum theses)
\def\StudyProgramme{\xxx{Mathematical modelling}}

% An optional dedication: you can thank whomever you wish (your supervisor,
% consultant, who provided you with tea and pizza, etc.)
\def\Dedication{%
\xxx{Dedication.}
}

% Abstract (recommended length around 80-200 words; this is not a copy of your thesis assignment!)
\def\Abstract{%
\xxx{Fluid-structure interactions (FSI) describe the interplay between fluids and
deformable structures, critical in fields like engineering and biomechanics.
Due to their complexity, FSI problems often require numerical simulations.
This paper adopts an approach using the Smoothed Particle Hydrodynamics (SPH) method.
We build on prior SPH research, based on the Symmetric Hyperbolic Thermodynamically Compatible equations (SHTC),
and validate the method
through tests on the Turek-Hron FSI benchmark.
}
}

% 3 to 5 keywords (recommended) separated by \sep
% Keywords are useful for indexing and searching for the theses by topic.
\def\ThesisKeywords{%
\xxx{keyword\sep key phrase}
}

% If any of your metadata strings contains TeX macros, you need to provide
% a plain-text version for use in XMP metadata embedded in the output PDF file.
% If you are not sure, check the generated thesis.xmpdata file.
\def\ThesisAuthorXMP{\ThesisAuthor}
\def\ThesisTitleXMP{\ThesisTitle}
\def\ThesisKeywordsXMP{\ThesisKeywords}
\def\AbstractXMP{\Abstract}

% If your abstracts are long and do not fit in the infopage, you can make the
% fonts a bit smaller by this setting. (Also, you should try to compress your abstract more.)
\def\InfoPageFont{}
%\def\InfoPageFont{\small}  % uncomment to decrease font size

% If you are studing in a Czech programme, you also need to provide metadata in Czech:
% (in English programmes, this is not used anywhere)

\def\ThesisTitleCS{\xxx{Modelování interakce tekutin a pevných látek pomocí Smoothed Particle Hydrodynamics}}
\def\DepartmentCS{\xxx{Matematický ústav UK}}
\def\DeptTypeCS{\xxx{Ústav}}
\def\SupervisorsDepartmentCS{\xxx{Matematický ústav UK}}
\def\StudyProgrammeCS{\xxx{Matematické momdelování}}

\def\ThesisKeywordsCS{%
\xxx{klíčová slova\sep klíčové fráze}
}

\def\AbstractCS{%
\xxx{Abstrakt práce přeložte také do češtiny.}
}
